\documentclass{journal}
\usepackage{graphics,graphicx}
\usepackage{hyperref}
\usepackage{fullpage}
\usepackage{pdflscape}
\usepackage{tikz}
\usepackage{gantt}
\usepackage{setspace}

\onehalfspacing

\begin{document}
\bibliographystyle{IEEE}
\nocite{*}
\title{Interactive Graduate Student Information Database}
\subtitle{Midterm Report} 
\author{Kartik Thakore (250313003)\\kthakore@uwo.ca \and Parth Champaneri (250367669)\\pchampan@uwo.ca}
\maketitle

\begin{abstract}
This report aims to provide preliminary elicited requirements for the Interactive Graduate Student Information System (SIMS). Additionally a walk through of the first iteration is included. We started work on the first iteration which includes two 2 critical features of SIMS. Data schemas and logic for Authentication and Student Funding were accomplished. Using the analysis of elicited requirements, data flow requirements, and verified assumptions a database schema was designed. After normalizing the database schema, a Representational State Transfer (REST) framework was implemented to combine components. With a preliminary implementation in place unit and integration testing were performed. A sign-off from end users and clients is required to prepare from the next iteration. However the components for the next iteration have been decided to include the E-signature component for the advisory meeting tracking feature. In preparation for iteration 2 we have begun rapid prototyping of the E-signature component.\end{abstract}

\section{Specific requirements}
\subsection{External interface requirements}
\subsubsection{User interfaces}
\subsubsection{Hardware interfaces}
\subsubsection{Software interfaces}
\subsubsection{Communications interfaces}
\subsection{System features}
\subsubsection{Graphical User Interface}
\begin{enumerate}
\item Purpose: Provide a dossier format interface for users to see all relevant data of the student in a centralized location.
\item Response sequence: User will login, and be able to search a student or go through a table of all available users. Selecting a student will bring the user onto this interface.
\item Associated Functional Requirements:
\begin{enumerate}
\item Drill down: Functional Requirement 1
\begin{enumerate}
\item Sections: each student will have multiple sections that can be disabled or enabled.
\item Expansion: sections will expand to show summary of addition information.
\item Link: sections will link to sections pages for more detailed information.
\end{enumerate}
\end{enumerate}
\end{enumerate}
\subsubsection{Term Calculations}
\begin{enumerate}
\item Purpose: Calculate dates and event times for graduate student programs 
\item Response sequence: When a student profile and program is created the system will make triggers for relevant events to each milestone.
\item Associated Functional Requirements:
\begin{enumerate}
\item Milestones: Functional Requirement 1
\begin{enumerate}
\item Calculate start - end  Semester dates for Graduate Students
\item Calculate due dates and triggers for milestones
\end{enumerate}
\item Funding Calculations: Functional Requirement 2
\begin{enumerate}
\item Track funding availability for each semester and the source of funding
\item Show next date for major funding applications
\end{enumerate}
\end{enumerate}
\end{enumerate}
\subsubsection{Tracking}
\begin{enumerate}
\item Purpose: Track Major Milestones (grants, publications, exams etc) for graduate students through the program.
\item Response sequence: Data is updated according to business rules and workflow of the program and the student's progress.
\item Associated Functional Requirements:
\begin{enumerate}
\item Publications and Grants: Functional Requirement 1
\begin{enumerate}
	\item Track student publications that have been published 
	\item Track grants that have been received by the student
\end{enumerate}
\item Advisory Committee Members: Functional Requirement 2
\begin{enumerate}
	\item Send trigger to student to form an Advisory Committee
	\item Allow students, and advisory committees to store and track comments and discussions
	\item Show calendar view of all meetings and results of the meetings
	\item Allow for single or joint supervisors 
	\item Track electronic submissions of advisory meeting form
\end{enumerate}
\item Manage Milestones: Functional Requirement 3
\begin{enumerate}
    \item Handle and process milestones for the Masters program in BioMedical Physics at UWO
    \begin{enumerate}
	\item Form advisory committee by end of 1st term
	\item Annual seminars 
	\item Low-level exams for new students
	\item Exams are organized by department
	\item Exams usually in late June; informed in early May
	\item Possible MSc to PhD reclassification
	\item Discuss reclassifications with supervisor and advisory committee first
	\item Reclassification must be completed before end of 5th semester
	\item Submit and defend MSc thesis if not reclassified
	\item \url{http://www.uwo.ca/biophysics/grad\_program\_policies/guidelines\_intro.htm}
    \end{enumerate}
\end{enumerate}
\item Send Triggers and Receive Responses: Functional Requirement 4
\begin{enumerate}
\item Process conditional and requested triggers
\item Allow Faculty Advisor to create and view all triggers
\item Conditional triggers are event based automatic or triggered conditions
\item Requested Triggers are created by users and their activities on the system
\item The system should allow responses to each Triggers be collected and stored
\item Responses should be accessible by relevant users only 
\end{enumerate}
\end{enumerate}
\end{enumerate}
\subsubsection{User Layers and Collaboration}
\begin{enumerate}
\item Purpose: Ensure ad-hoc access for multiple users to facilitate realtime collaboration and ensure up-to date information in the database. Permissions map to prevent unauthorized access and control the scope of data.
\item Response Sequence: Multiple users will be able to log in simultaneously and information will update in realtime.
\item Associated Functional Requirements:
\begin{enumerate}
\item User Group Permission Map
\begin{enumerate}
\item Graduate Students: This group of users will be able to log in and able to edit, update and save their demographic information and other program information including Advisory Committee members, Publications, Thesis etc.
\item Advisory Committee Members: This group will be able to comment  and provide feedback on a student's advisory committee meeting output. Ideally, other student information will be restricted for changes.
\item Graduate Executives: This group of users will primarily utilize the generated reports for planning and information purposes. Access will be restricted to viewing information. Will be taken to a Project dashboard where they will have a bird's eye view of reports and information statistics. Read-Only Access.
\item Graduate Affairs Assistant: Key stakeholder for the system. WIll be able to manage, administer and access all informational program data. Access to change log. Generate reports in Excel, PDF etc.
\item Technical Administrator: Primarily responsible for system administration, periodic maintenance schedule and providing technical assistance to users. Ability to reset system passwords and create users.
\end{enumerate}
\end{enumerate}
\end{enumerate}

\subsubsection{Security}
\begin{enumerate}
\item Purpose: To build a secure system that adheres to local and federal privacy laws (FIPPA)
\item Response Sequence: Any data inputted into the system will be encrypted and all passwords will be stored using hash.
\item Associated Functional Requirements:
\begin{enumerate}
\item To be defined
\end{enumerate}
\end{enumerate}


\subsubsection{Database design}
\begin{enumerate}
\item Purpose: Improve Error detection and stability of the system.
\item Response Sequence: The information will be normalized and data redundancy will be introduced
\item Associated Functional Requirements:
\begin{enumerate}
\item Data Normalization: Systematic way to ensure that the design is free from any undesirable characteristic - insertion, update, and deletion anomalies that could lead to the loss of data integrity.
\item Data Redundancy to improve error detection
\end{enumerate}
\end{enumerate}
\subsection{Performance requirements}

\subsection{Design constraints}
\subsection{Software system attributes}
\subsection{Other requirement}

\newpage

\section{Iteration 1}
\subsection{Features}
\subsubsection{Authentication}
Multiple user access to the systems from an organization role aspect.
\subsubsection{Student Funding}
Feature to archive student funding data, calculate ending term and provide reports.
\subsection{Analysis}
\subsubsection{Roles and Operation Analysis}
Organizational roles and what need 
\subsubsection{Data Flow Analysis}
What are the criteria for our DFD components. Concerns .. 
\subsubsection{Relationship Analysis}
Critical Assumptions, Scope checking
\subsection{Design}
\subsubsection{Use Class Diagrams}
What can a user do
\subsubsection{Access Control List}
Who can use what
\subsubsection{Data Flow Diagrams}
Sources, processes, Sinks
\subsubsection{Class Diagrams}
Auth.PNG and StudentFunding.PNG
\subsection{Implementation}
\subsubsection{Catalyst Framework}
Chained Operations
\subsubsection{Rapid Database Prototyping}
SQLite database and rapid schema changing
\subsection{Testing}
\subsubsection{Regression Testing}
Unit Tests
\subsubsection{Acceptance Testing}
User Tests and feedback

\section{Iteration 2}
Add Student milestone tracking.
\subsection{Rapid Prototyping}
For signature Pad
\section{Progress}
\subsection{Updated Gantt Chart}
\subsection{Changes}
\bibliography{ref}

\end{document}



