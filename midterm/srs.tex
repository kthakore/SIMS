\documentclass{journal}
\usepackage{graphics,graphicx}
\usepackage{hyperref}
\usepackage{fullpage}
\usepackage{pdflscape}
\usepackage{tikz}
\usepackage{gantt}
\usepackage{setspace}
\usepackage{enumitem}
\usepackage{index}
\makeindex
\onehalfspacing

\begin{document}

\bibliographystyle{IEEE}
\nocite{*}
\title{Interactive Graduate Student Information Database}
\subtitle{Midterm Report} 
\author{Kartik Thakore (250313003)\\kthakore@uwo.ca}
\maketitle



1
Introduction
1.1
Purpose
The purpose of this report is to document preliminary elicited requirements for the Interactive Gradu-
ate Student Information System (SIMS). Additionally, in accordance with the Agile project management
methodology, a walk through of the first iteration is documented. Our first iteration includes two system
critical features which will be used for iterative system development of the system. Database schemas and
operational logic for user authentication and calculations for graduate student funding were accomplished.
A database schematic was designed after performing extensive analysis based on our specifications, data flow
requirements and approved system critical assumptions. Data normalization was performed and a Represen-
tational State Transfer (REST) framework was implemented to combine components. Furthermore, unit and
integration testing were performed with the preliminary implementation in place. For preparation of future
iterations, a sign-off from end users and client is required. However, we have started the rapid prototyping
of the e-signature component for the advisory meeting tracking feature.
1.2
Scope
The scope of the following requirements is to provide a pilot software back end for the BioMedical Physics
Graduate program at UWO. First the system will replace the current method of tracking student information
pictured in 1.
• The current system provides the following features which directly fall into scope for this product:
– Storage of Intrinsic Student Value
– Graphical User Interface for Entering Student Data
– Saving portable reports for external users
• The new system will additionally need to address the following concerns:
– Store data securely and with more redundancy
– Simply the User Interface for using the system
– Remove the load of managing data solely on one user
– Allow faculty members to view data in a meaningful way
– Allow students to update their information
– Allow advisory committee members to store and retrieve signatures and comments on the system
Currently the scope is limited to tackling these core issues. However since the project cycle is iterative,
the scope can be expanded in a controlled manner, should the situation demand it.
1.3
Definitions, acronyms, and abbreviations
• SIMS: Student Information System
• UWO: University of Western Ontario
• REST: Software Architecture for Web Applications
• VPN: Virtual Private Network
• SSL: Security Protocol for Web Applications
2
Fig. 1: The Current System
1.4 References
1.5 Overview
The SIMS core is a secure application that aims to be flexible to handle business rules of varying graduate
student programs. In the scope of this project however, focus will be placed on the Graduate Student
Program at the BioMedical Physics program at University of Western Ontario. At the very least SIMS
will hold graduate student, funding and advisory meeting data. Additionally SIMS will allow the program
administrator to organize collected data into reports and to send automatic request for data to students.
Also advisors and other faculty users will be able to see student data, progress and any advisory meetings
they have attended. SIMS will also place an emphasis on providing security of student personal data and
faculty identifications. Overall SIMS, will be replacing the current manual system of tracking students and
their progress through graduate programs.
Figure 2 shows the overview of the required system. The user will access the system via the Client and
Tablet devices. The data from the tablet will be processed in the client and sent via the TCP/IP protocol
to the Application Server (AppServer). The responsibility of the AppServer is to provide secure access, and
host the application. The AppServer will communicate with the Services Server to add triggers and use the
database. The AppServer and the ServicesServer will be located on a local network which is accessed via
a Virtual Private Network (VPN). Figure 3 models the users of the SIMS system. The user groups can be
broken down into 3 categories, the Student, the Faculty and the Technical Administrator. The technical
administrator will have access only to the authentication data. While the student and the faculty will have
access only to the tracked data. Additionally the faculty will have more access over the student. The role
and operations requirements will be covered more in-depth in the system features.
3
Fig. 2: The proposed system
Fig. 3: Users of the SIMS system
4
2
Overall Description
2.1
Product perspective
The product will be self contained as it is responsible form User Interface, Application and Data Storage.
Figure 2 again describes the perspective of the system.
The product will have the following constraints:
• User Interface: The specifications of each view of the user interface will be provided for this project.
• Hardware Interface: The system will be using a Wacom tablet as a hardware to acquire signatures.
• Software interfaces: The system will have to employ several software interfaces.
– Application Server: Apache 2.0 will be used to deploy the system with SSL security.
– VPN Provider: OpenVPN will be used to deploy our system as a self contained virtual private
network.
– Database Software: A PostgreSQL RDBMS server will be used to run our Data Storage Compo-
nent of the Software.
2.2 Product function
2.3 User characteristics
2.4 Constraints
2.5 Assumptions and dependencies
3
Specific Requirements
3.1 External interface requirements
3.2 Functional requirements
3.3 Performance requirements
3.4 Design constraints
3.5 Software system attributes
3.6 Other requirements
4 Appendixes
5 Index

