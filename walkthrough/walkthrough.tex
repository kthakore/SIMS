% $Header: /cvsroot/latex-beamer/latex-beamer/solutions/generic-talks/generic-ornate-15min-45min.en.tex,v 1.5 2007/01/28 20:48:23 tantau Exp $

\documentclass{beamer}

% This file is a solution template for:

% - Giving a talk on some subject.
% - The talk is between 15min and 45min long.
% - Style is ornate.



% Copyright 2004 by Till Tantau <tantau@users.sourceforge.net>.
%
% In principle, this file can be redistributed and/or modified under
% the terms of the GNU Public License, version 2.
%
% However, this file is supposed to be a template to be modified
% for your own needs. For this reason, if you use this file as a
% template and not specifically distribute it as part of a another
% package/program, I grant the extra permission to freely copy and
% modify this file as you see fit and even to delete this copyright
% notice. 


\mode<presentation>
{
  \usetheme{Warsaw}
  % or ...

  \setbeamercovered{transparent}
  % or whatever (possibly just delete it)
}


\usepackage[english]{babel}
% or whatever

\usepackage[latin1]{inputenc}
% or whatever

\usepackage{times}
\usepackage[T1]{fontenc}
% Or whatever. Note that the encoding and the font should match. If T1
% does not look nice, try deleting the line with the fontenc.


\title[Project Title] % (optional, use only with long paper titles)
{Graduate Student Information System (gSIMS) }

\subtitle
{Walkthrough} % (optional)

\author[Author] % (optional, use only with lots of authors)
{Kartik Thakore\inst{1}}
% - Use the \inst{?} command only if the authors have different
%   affiliation.

\institute[University Information] % (optional, but mostly needed)
{
  \inst{1}%
  Department of Software Engineering\\
  University of Western Ontario
}
% - Use the \inst command only if there are several affiliations.
% - Keep it simple, no one is interested in your street address.

\date[Date] % (optional)
{23 Nov 2010}

\subject{Talks}
% This is only inserted into the PDF information catalog. Can be left
% out. 



% If you have a file called "university-logo-filename.xxx", where xxx
% is a graphic format that can be processed by latex or pdflatex,
% resp., then you can add a logo as follows:

\pgfdeclareimage[height=0.5cm]{university-logo}{university-logo-filename}
\logo{\pgfuseimage{university-logo}}



% Delete this, if you do not want the table of contents to pop up at
% the beginning of each subsection:
\AtBeginSubsection[]
{
  \begin{frame}<beamer>{Outline}
    \tableofcontents[currentsection,currentsubsection]
  \end{frame}
}


% If you wish to uncover everything in a step-wise fashion, uncomment
% the following command: 

%\beamerdefaultoverlayspecification{<+->}


\begin{document}

\begin{frame}
  \titlepage
\end{frame}

\begin{frame}{Outline}
  \tableofcontents
  % You might wish to add the option [pausesections]
\end{frame}


% Since this a solution template for a generic talk, very little can
% be said about how it should be structured. However, the talk length
% of between 15min and 45min and the theme suggest that you stick to
% the following rules:  

% - Exactly two or three sections (other than the summary).
% - At *most* three subsections per section.
% - Talk about 30s to 2min per frame. So there should be between about
%   15 and 30 frames, all told.

\section{Introduction}

\subsection[Project Details]

\begin{frame}{Project Inception}
  % - A title should summarize the slide in an understandable fashion
  %   for anyone how does not follow everything on the slide itself.

  \begin{itemize}
  \item
    Use \texttt{itemize} a lot.
  \item
    Use very short sentences or short phrases.
  \end{itemize}
\end{frame}

\begin{frame}{The Team}

  You can create overlays\dots
  \begin{itemize}
  \item using the \texttt{pause} command:
    \begin{itemize}
    \item
      First item.
      \pause
    \item    
      Second item.
    \end{itemize}
  \item
    using overlay specifications:
    \begin{itemize}
    \item<3->
      First item.
    \item<4->
      Second item.
    \end{itemize}
  \item
    using the general \texttt{uncover} command:
    \begin{itemize}
      \uncover<5->{\item
        First item.}
      \uncover<6->{\item
        Second item.}
    \end{itemize}
  \end{itemize}
\end{frame}

\begin{frame}{Problem Definition}
\end{frame}

\begin{frame}{Scope}
\end{frame}


\subsection{Requirements}

\begin{frame}{Interfaces}
\end{frame}

\begin{frame}{Graphical User Interface}
\end{frame}


\begin{frame}{Electrical Device Interface}
\end{frame}

\begin{frame}{System Features}
\end{frame}

\begin{frame}{Constraints}
\end{frame}

\subsection{Analysis}
\begin{frame}{Oranizing Data}
\end{frame}
\begin{frame}{Conceptual Model of the Student}
\end{frame}
\begin{frame}{Critical Assumptions}
\end{frame}

\subsection{Architecture}
\begin{frame}{Hardware}
\end{frame}
\begin{frame}{Software}
\end{frame}
\begin{frame}{Network Protocols and Schemes}
\end{frame}
\begin{frame}{ REST Web Applications}
\end{frame}
\begin{frame}{Perl Batch Services}
\end{frame}

\subsection{Iteration 1}
\begin{frame}{System Features}
\end{frame}
\begin{frame}{Intrinstic Data of a Student}
\end{frame}
\begin{frame}{Role Based Authentication}
\end{frame}

\subsection{Iteration 2}
\begin{frame}{System Features}
\end{frame}
\begin{frame}{E-Signature Clien}
\end{frame}

\subsection{ Test Plans}
\begin{frame}{Unit Tests}
\end{frame}
\begin{frame}{Integration Testing}
\end{frame}
\begin{frame}{System Integration Testing}
\end{frame}





\section*{Summary}

\begin{frame}{Summary}

  % Keep the summary *very short*.
  \begin{itemize}
  \item
    Requirements and Analysis has received direct user feedback.
  \item
    Architecture based of the Analysis has been clarified and prototyped. 
  \item 
	The iterative Software Life Cycle has produced useful work quickly and with less effort.
  \item
    A strong emphasis on 3 testing levels is present from the starting.
  \end{itemize}
 \end{frame}


\end{document}


